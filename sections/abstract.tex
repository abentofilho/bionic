\begin{abstract}
	Foot is the only human body part that direct contacts the ground during walking with important role in human locomotion and balance. In a foot amputee person, a trans-tibial prostheses aim to substitute the foot, tibia and fibula. the prostheses can be passive or active. Passive prostheses only supports amputee's movement, demanding higher metabolic costs, not providing active functions, like the ankle's propulsion in the displacement, nor providing the absorption of impacts. Active prostheses, on the other hand, have motors, brakes and dampers which improve the mobility by supplying energy, applying brakes and allow damping when needed. 
	\textcolor{red}{  These prostheses facilitate simple movement and execution of more complex daily activities, like w. Due to the characteristics presented, many researches have been dedicated to developing active prostheses more suitable for the user. The challenge is to develop a device with power and speed of action suitable for the function in a light and compact structure. Thus, this work aims to develop a digital prototype of a bionic foot for transtibial amputees. The load and movement capacity of a healthy foot were used to size the components of the prosthetic foot for a medium-sized person. The properties of the prototype, dimensions, mass and inertia were used to construct a dynamic model for simulations and implementation of a controller. The developed prototype will be used to manufacture a physical prototype in future works.}\\
	xxxx
	Insert your abstract here. Include keywords, PACS and mathematical
	subject classification numbers as needed.
	\keywords{First keyword \and Second keyword \and More}
	% \PACS{PACS code1 \and PACS code2 \and more}
	% \subclass{MSC code1 \and MSC code2 \and more}
\end{abstract}
